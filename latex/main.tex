\documentclass[twoside]{article}
\usepackage{preamble_style}
\usepackage{preamble}
\usepackage{bibliography}


\title{Numerical study of the 2D Kuramoto-Sivashinsky equation}
\author{Víctor Ballester}
\date{\parbox{\linewidth}{\centering
Instabilities and nonlinear phenomena\endgraf
M2 - Applied and Theoretical Mathematics\endgraf
Université Paris-Dauphine \& PSL\endgraf
\today}}


\begin{document}
\maketitle
\begin{abstract}
  This is an abstract.
\end{abstract}
{
\hypersetup{linkcolor=black}
\tableofcontents
}

\section{Introduction}
2d KS equation:
\begin{equation}
  \begin{cases}
    u_t = -\frac{1}{2}\abs{\grad u}^2 - \laplacian u - \laplacian^2 u & \text{in } (0, \infty) \times [0, L_x) \times [0, L_y) \\
    u(t, x, y) = u(t, x + L_x, y)                                     & \text{in } [0, \infty) \times \RR \times [0, L_y)      \\
    u(t, x, y) = u(t, x, y + L_y)                                     & \text{in } [0, \infty) \times [0, L_x) \times \RR      \\
    u(0, x, y) = u_0(x, y)                                            & \text{for all } x \in [0, L_x), y \in [0, L_y)
  \end{cases}
\end{equation}
If we rescale the spatial domain to $[0, 2\pi)^2$ and the time domain using the transformations
\begin{equation}
  x_\mathrm{new} = \frac{2\pi}{L_x} x \qquad y_\mathrm{new} = \frac{2\pi}{L_y} y \qquad t_\mathrm{new} = {\left(\frac{L_x}{2 \pi}\right)}^2 t
\end{equation}
we get the following equation
\begin{equation}
  u_t=-\frac{1}{2}\abs{\grad_\nu u}^2-\laplacian_\nu u-\nu_1{\laplacian_\nu}^2 u
\end{equation}
where we used a slightly-modified notation from the one in \cite{Kalogirou2015}
\begin{align}
  \grad_\nu & = \left(\partial_x , \sqrt{\frac{\nu_2}{\nu_1}} \partial_y
  \right)   & \div_\nu                                                   & = \partial_x + \sqrt{\frac{\nu_1}{\nu_2}} \partial_y
  \\ \laplacian_\nu &= \div_\nu(\grad_\nu) = \partial_{xx} + \frac{\nu_2}{\nu_1} \partial_{yy} & {\laplacian_\nu}^2 &= \laplacian_\nu(\laplacian_\nu) = \partial_{xxxx} + 2 \frac{\nu_2}{\nu_1} \partial_{xxyy} + \frac{\nu_2^2}{\nu_1^2} \partial_{yyyy}
\end{align}
and $\displaystyle\nu_1 ={\left( \frac{L_x}{2\pi} \right)}^2$, $\displaystyle\nu_2 = {\left( \frac{L_y}{2\pi} \right)}^2$ and we have dropped the subindices \emph{new} for simplicity.

Note that the equation is invariant under the transformation $(t,x,y, \nu_1, \nu_2) \mapsto \left( \frac{\nu_2}{\nu_1} t, y, x, \nu_2, \nu_1 \right)$. That is, if $u(t,x,y)$ is a solution of the equation with parameters $(\nu_1, \nu_2)$, then $u\left( \frac{\nu_2}{\nu_1} t, y, x \right)$ is the solution of the equation for the parameters $(\nu_2, \nu_1)$.

\phantomsection
\addcontentsline{toc}{section}{References}
\printbibliography
% Add a new empty sheet at the end of the document
\end{document}













